\documentclass[12pt,a4paper]{article}
\usepackage{times}
\usepackage{durhampaper}
\usepackage{harvard}
\citationmode{abbr}
\bibliographystyle{agsm}

\title{RAWFlash: A cloud based RAW image editor}
\author{} % leave; your name goes into \student{}
\student{Ryan Collins}
\supervisor{Dr Tom Friedetzky}
\degree{[MEng Computer Science]}

\date{\today}

\begin{document}

\maketitle

\begin{abstract}
% These instructions give you guidelines for preparing the final paper.  DO NOT change any settings, such as margins and font sizes.  Just use this as a template and modify the contents into your final paper.  Do not cite references in the abstract.

% The abstract must be a Structured Abstract with the headings {\bf Context/Background}, {\bf Aims}, {\bf Method}, {\bf Results}, and {\bf Conclusions}.  This section should not be longer than half of a page, and having no more than one or two sentences under each heading is advised.
\paragraph{Context/Background}

\paragraph{Aims}

\paragraph{Method}

\paragraph{Results}

\paragraph{Conclusions}

\end{abstract}

\begin{keywords}
Put a few keywords here.
\end{keywords}

\section{Introduction}
% What is the project about?

% Context of the Project

\subsection{Applications of RAW Image Editing}


\subsection{Problems with Current Implementations}
% Native: specific software
% : can't be employed easily in whatever environment needed.


\subsection{Project Objectives}

% Develop a Cloud-based RAW Image Photo Editor. 
\section{Related Work}
% This section presents a survey of existing work on the problems that this project addresses.  it should be between 2 to 4 pages in length.  The rest of this section shows the formats of subsections as well as some general formatting information for tables, figures, references and equations.


\subsection{RAW files, and parsing}
% DNG standard
% 
\subsection{Existing Solutions}
% Lightroom, proprietary
% Darktable, RAW Therapee
% DCRAW, LibRAW
% ImageMagick => not as useful for RAW though.

\subsection{}


\section{Solution}

% This section presents the solutions to the problems in detail.  The design and implementation details should all be placed in this section.  You may create a number of subsections, each focussing on one issue.  

% This section should be between 4 to 7 pages in length.

\subsection{Client-Server Communication Protocol}
% SocketIO
% REST
% benefits
% transferring images: TIFF, PNG, base64 vs file based.
% compatibility.


\subsection{Rendering Server Structure}
% Layers
% Adams processor vs DCRAW
% Message Queue vs full Socket IO (drawbacks of using SocketIO on its own)
% Implementing gamma correction (algorithms used)
% Implementing Colour adjustments (algorithms used)

\subsection{Web Editor Functionality}
% While render server does all the rendering, basic functionality of an editor should be present in the example
% cloud editor, for  testing.

% Implementing an Undo system: algorithms, data storage.
% User interface: sidebar with less cluttered nature.

% 
\subsection{Storing User Images}
% While editing individual images is ok, editing a selection of
% images would be more functional, allowing us to select server uploaded
% images, edit them, and export them.

\subsection{Managing a big system}
% Docker/DockerCOmpose for managing a massive distributed system.
% Docker automatic restart on error 
\subsection{Unit Testing Issues with Image Processing}
% Can't just expect a given output, because each algorithm may change the image in a different way,
% and in some cases we are using floating point to do image adjustments. This causes some error, and therefore
% we won't necessarily get exactly the same image out.
\section{Results}

% this section presents the results of the solutions.  It should include information on experimental settings.  The results should demonstrate the claimed benefits/disadvantages of the proposed solutions.

% This section should be between 2 to 3 pages in length.

\subsection{Comparison With Other Editors}
Testing the system against Lightroom, Darktable and RawTherapee,
testing the image output compared to these, and also performance
based tests compared with traditional native software.

TODO: Mention specs of the machine.


\subsection{User Testing}
Get User Comments on system

\subsection{}

\section{Evaluation}

% This section should between 1 to 2 pages in length.

\subsection{Application to Web Content Management Systems}
% Process of editing and uploading images
% RAW -> JPEG -> Upload -> Further compression/adjustments in CMS for web -> Output
% better to skip intermediate steps, meaning less computational effort on one end, less cost for
% expensive software.

\subsection{Benefits of Portable Image Editing}
% Allows one to edit images whatever the location,

\subsection{Performance Issues and Improvements}
% Improvement by use of multithreading for image processing
% Java's build in Image Manipulation has a number of undocumented bugs.
% Using GPU might improve performance furhter.


\section{Conclusions}

This section summarises the main points of this paper.  Do not replicate the abstract as the conclusion.  A conclusion might elaborate on the importance of the work or suggest applications and extensions.  This section should be no more than 1 page in length.

The page lengths given for each section are indicative and will vary from project to project but should not exceed the upper limit.  A summary is shown in Table \ref{summary}.

\begin{table}[htb]
\centering
\caption{SUMMARY OF PAGE LENGTHS FOR SECTIONS}
\vspace*{6pt}
\label{summary}
\begin{tabular}{|ll|c|} \hline
& \multicolumn{1}{c|}{\bf Section} & {\bf Number of Pages} \\ \hline
I. & Introduction & 2--3 \\ \hline
II. & Related Work & 2--3 \\ \hline
III. & Solution & 4--7 \\ \hline
IV. & Results & 2--3 \\ \hline
V. & Evaluation & 1-2 \\ \hline
VI. & Conclusions & 1 \\ \hline
\end{tabular}
\end{table}


\bibliography{projectpaper}


\end{document}