\title{Literature Review: Cloud-based RAW Image editor}
\author{
        Ryan Collins\\
        Supervisor: Tom Friedetzky
}
\date{\today}
\documentclass[14pt]{article}
\usepackage[utf8]{inputenc}
\usepackage[english]{babel}
\usepackage{comment}
\usepackage[
  backend=biber,
  style=alphabetic,
  sorting=ynt
]{biblatex}
\addbibresource{lit.bib}
\begin{document}
\maketitle

% \begin{abstract}
% This is the paper's abstract \ldots
% \end{abstract}

\section{Introduction}
\paragraph{Problem Background}
Typically, RAW image editors require the user to have a fairly high specification machine
to process the RAW image files. This project aims to create a system that is cloud-based,
requiring only a web browser on the client machine to edit RAW images. The benefit is also
one of interlectual property, as the RAW image file is not necessarily stored locally,
preventing the RAW asset from being stolen, useful in companies where the image files must
be secure.

\paragraph{Areas of Research}
There are various areas to research in order to ensure the project is successful. These are:

\begin{itemize}
  \item What is the RAW image format, and how does it differ from other formats such as JPEG?
  \item What are the current 
\end{itemize}

Each of these questions needs answering, in order for the system to be developed fully.

\section{Definitions}
\begin{tabular}{| c | c |}
    Term & Definition \\
    \hline
    RAW & A RAW file is a collection of unprocessed data. In the case of cameras, the data stored
          is uncompressed, and the RAW camera sensor data is stored, with values for each brightness in the bayer matrix, for each pixel.
\end{tabular}

\section{Important Issues of Identified Themes}
% This is the meat and potatoes. We need to answer each question proposed under the 'areas of research' section

\section{Proposed Direction of Project}
% Here is where we specify how we proceed with the project

\section{Conclusion}
% The conclusion.
\section{References}\label{references}
\printbibliography
\end{document}
